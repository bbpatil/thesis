\chapter{Emulation and Hardware in the loop}
\label{cha:emulation}

The field of emulation and \emph{HiL} is strongly connected to the field of simulation.
Emulation and \emph{HiL} includes the connection of an simulated system to the real world.
Such real world systems can either be a software (software in the loop \emph{SiL}) or real hardware components (\emph{HiL}).
This connection of simulated systems with real systems can be used for testing of those systems.
Such a test method can test various different scenarios due to the flexibility of the simulated system.

The most essential part for the field of emulation and \emph{HiL} is the connection possibilities of the simulation.
I.e. the part of the simulation which is communicating with the real world and creates simulation events for occasions from the real world.
%TODO reference


\section{Emulation with OMneT++}
\label{sec:emulation_omnet}
For the fields of emulation and \emph{HiL} OMNeT++ provides the sample simulation \emph{sockets} for showing the possibilities and methods for developing an emulation.

\subsection{Existing functionality}
\label{sec:emulation_omnet_existing}
As mentioned above the OMNeT++ sample \emph{sockets} demonstrates a possibility for the implementation of a communication interface to the real world.
The used custom scheduler \emph{SocketRTScheduler} listens to the network interface during wait times.
This behavior is usable for this demographic usage, but if the timings of the simulated systems sharpen this scheduler would no allow sufficient communication.


\section{Communication with the real world}
The communication module can be represented by two separate simple modules which implement the whole connection to the real world separated to receiving and sending.

\begin{description}
    \item[Receiving] The receiving module can be implemented using the \emph{process style} strategy described in section \ref{sec:omnet_components_modules}.
    By using this strategy the module can listen to the communication interface and create simulation events for external occasions.
    Executing the simulation sequentially does not allow constant listening by such a receiver module, therefore this must be interrupted to allow the execution of the remaining simulation.
    For this execution method the scheduler \emph{SocketRTScheduler} used in the \emph{sockets} sample provides a more optimized implementation strategy.
    
    Using parallel simulation described in chapter \ref{cha:parallel_sim} the receiving module can be implemented blocking.
    Assigning just the receiving module to a single processor allows a constant listening to the communication interface.
    This represents an improved behavior and extended possibility for handling real time occasions.
    
    \item[Sending] The sending module can be implemented independently of the used execution method, except the sending to the communication interface is implemented blocking.
    If the blocking is inevitable this module either should be implemented using the \emph{process style} strategy and be executed on a separate processor or the sending must be executed with care to non blocking behavior.
    Such behavior could be accomplished with timeouts and retries after a defined waiting time while the simulation can proceed.
\end{description}