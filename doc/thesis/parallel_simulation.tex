\chapter{Parallel simulation}
\label{cha:parallel_sim}
The so called sequential simulation is executed on a single processor, either logical, physical processor or distributed machines.
Simulating complex models and a huge number of events this simulation can take long time to execute.
The distributed discrete event simulation (\emph{DDES}) or parallel discrete event simulation (\emph{PDES}) allows the parallel execution on multiple processors.

Running a \emph{OMNeT++} simulation using multiple processors different requirements must be met.

\section{Parallel simulation with OMNeT++}
\label{sec:parallel_omnet}

For a parallel simulation using OMNeT++ different implementations for communication and synchronization must be defined.

\subsection{Communication}
\label{sec:parallel_omnet_comm}

a message passing interface (\emph{MPI}) library must be provided.
Rhe open source library openMPI can be used easily linux operating systems.


\subsection{Sychronisation}
\label{sec:parallel_omnet_sync}

% imported from paper
As described in \cite{varga_parallel_2003} OMNeT++ is capable of running a parallel simulation when specific requirements are met.
In \cite{varga_parallel_2003} the requirements include the compliance to OMNeT++ design guides, for example the strict usage of messages and channels for transmitting events and data.
Is this requirement not met and communication between modules was achieved by a direct method call the simulation cannot be executed parallel.
Running the simulation parallelized can achieve better timings and improve the performance of real time simulations
This improvement could be possible due to shortened execution times and also the extended possibility for emulation and the field of \emph{HiL}.
For this usage a module can be assigned to a logical processor and handle all messages which interfere with connected real systems.