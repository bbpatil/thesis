\chapter{Parallel simulation}
\label{cha:parallel_sim}
The so called sequential simulation is executed on a single processor, either logical, physical processor or distributed machines.
Simulating complex models and a huge number of events this simulation can take long time to execute.
The distributed discrete event simulation (\emph{DDES}) or parallel discrete event simulation (\emph{PDES}) allows the parallel execution on multiple processors.


\section{Synchronization}
\label{sec:parallel_synchronization}
Running simulations in parallel requires a synchronization in between the logical processes (\emph{LP}) for synchronizing the processed simulation time on the different \emph{LPs}.
This simulation time can diverge very strongly due to the different set of processed events.
With varying event intervals this difference results in differencing steps within simulation time.
For the synchronization of parallel \emph{LPs} different algorithms for this synchronization are available, the choice of the used synchronization method affects the achievable performance of the \emph{PDES}.

Generally there are two types of synchronization algorithms:

\begin{enumerate}
    \item Optimistic algorithms allows \emph{LPs} to process events that may be timed in the future and roll back in the case of an event from the past.
    \item Conservative algorithms strictly forbid the processing of events until the guarantee can be given that no event will be scheduled before such events. \cite[chapter 2]{bagrodia_performance_2000}
\end{enumerate}

An example for a conservative algorithm is the \emph{Null Message Algorithm} (\emph{NMA}). \cite[section 2.1]{bagrodia_performance_2000} \cite{Varga03apractical}.

\section{Parallel simulation with OMNeT++}
\label{sec:parallel_omnet}

For a parallel simulation using OMNeT++ different implementations for communication and synchronization must be defined and specific requirements must be met.
These factors are discussed in the following sections.

\subsection{Communication}
\label{sec:parallel_omnet_comm}

a message passing interface (\emph{MPI}) library must be provided.
The open source library openMPI can be used easily linux operating systems.


\subsection{Sychronization}
\label{sec:parallel_omnet_sync}

\section{Requriements}
\label{sec:parallel_omnet_requirements}

% imported from paper
As described in \cite{varga_parallel_2003} OMNeT++ is capable of running a parallel simulation when specific requirements are met.
In \cite{varga_parallel_2003} the requirements include the compliance to OMNeT++ design guides, for example the strict usage of messages and channels for transmitting events and data.
Is this requirement not met and communication between modules was achieved by a direct method call the simulation cannot be executed parallel.
Running the simulation parallelized can achieve better timings and improve the performance of real time simulations
This improvement could be possible due to shortened execution times and also the extended possibility for emulation and the field of \emph{HiL}.
For this usage a module can be assigned to a logical processor and handle all messages which interfere with connected real systems.