\chapter{Design}
\label{cha:design}
Implementing a simulation can be done in various designs which are tailored for specific requirements.
Given an existing system different design questions regarding complexity of modules and the designed hierarchy are brought up.
The following section will discuss two fundamental design strategies and their impact on development and achievable results.
Furthermore the according functionalities and strategies within OMNeT++ are explained.
An according example simulation and the resulting performances are shown and discussed in chapter \ref{cha:measurements}.

\section{Modular design}
\label{sec:design_modular}
A modular design imply a bigger number of different components communicating with each other and thereby representing a defined functional unit.
This approach can lead to a more dynamic simulation and increase the reusability of different components and modules.
Developing a complex systems can also be eased by using a modular design and therefore an increased separation in smaller functional units.

A modular design can also provide more insight in a given system and the processed procedures.
This increased insight can be used for educational and demographic usages showing detailed informations about the internal procedures of a functional unit.
During development of the simulation or the simulated system a modular design can also provide more insight and therefore analyzing and debugging possibilities.
The increased number of simulated components and the separation of functionality compared to a monolithic design results in increased communication in between these components.

Designing a modular system in OMNeT++ is done by implementing a bigger number of modules connected with channels and grouped in compound modules.
A simulated network consisting of multiple modules and therefore implementing a modular design result in increased communication, i.e. message allocation, transmission and deallocation.
These additional procedures can strongly influence the achievable performance and must be analyzed carefully.

A example of a modular design using OMNeT++ components is shown in figure \ref{fig:OMNeTModularDesign}.
This example contains multiple compound modules consisting of enclosed simple and compound modules.
The multiple connections in between the modules show the required communication.

\begin{figure}
    \centering
    \includegraphics[width=0.9\columnwidth]{OMNeTModularDesign.eps}
    \caption{Modular design with OMNeT++}
    \label{fig:OMNeTModularDesign}
\end{figure}

Approaching emulation and the fields of \emph{HiL} using real-time simulation these design decisions are important for the achieved timings.

As described in section \ref{sec:parallel_omnet} OMNeT++ provides functionalities for running a parallelized simulation and therefore increase the performance by parallelism.
This type of executing is depending strongly on the simulated system and the applied design.
Due to the necessary communication in between parallel simulated modules a modular design provides more possibilities for parallelism than a monolithic design.

Developing a simulation for a given system with an existing implementation results in restraints for the applicable design.
The simulation of a system which is implemented modular can also be designed modular using the separations of the given system.
For a modular designed simulation the connection of simulated system and according simulation modules must be possible.
This connection can be achieved when the existing system is already using interfaces or function pointer for the connections of different components.
These communication parts can be used for redirecting calls to wrapper modules embedding the simulated system in the simulation environment.
These modules handle all received and create necessary outgoing messages.

\section{Monolithic design}
\label{sec:design_monolithic}
The opposite design to the modular design discussed in the previous section is the monolithic design.
A monolithic design imply a lower number of components, but an increased functionality within a single one.
This decreased number of components lower the necessary communication and a potential overhead.
The reduced modularity does not allow deep insight in the system but can result in improved performance.
Is a simulated system used as a single functional unit and maybe event instantiated multiple times within the simulation a monolithic design is favorable, especially when no detailed information about internal procedures is demanded.

Within OMNeT++ a monolithic design can be achieved by condensing a compound module to a simple module with more complex functions.
The execution of the single modules include more normal C++ code using simple method calls and operations.
%TODO enhance

The previous example network showing a modular design in figure \ref{fig:OMNeTModularDesign} can be condensed to a more monolithic design.
In the example, shown in figure \ref{fig:OMNeTMonolithicDesign}, the compound modules 0 and 2 with all their submodules were replaced by two simple modules.
The channels and the sent messages between the modules stay the same, but the calculation within the modules include the complete behavior of the previously included components.

\begin{figure}
    \centering
    \includegraphics[width=0.9\columnwidth]{OMNeTMonolithicDesign.eps}
    \caption{Monolithic design with OMNeT++}
    \label{fig:OMNeTMonolithicDesign}
\end{figure}

