\chapter{Simulation of openPOWERLINK}
\label{cha:porting}
The development of an OMNeT++ simulation including a POWERLINK network consisting of multiple nodes was achieved using the openPOWERLINK implementation.

For developing a simulation embedding an openPOWERLINK node, the openPOWERLINK stack must be ported within the OMNeT++ environment.
Therefore the platform dependencies as discussed in section \ref{sec:oplk_platform} are analyzed and additional simulation specific implementations are introduced.
These simulation specific implementations are described in the following section.

\section{Simulation stub}
\label{sec:porting_simstub}

The intention is to separate the simulation specific implementations and changes within the openPOWERLINK stack and the OMNeT++ implementations providing the simulation environment.
This was achieved by the introduction of a simulation stub in the openPOWERLINK stack.
This stub should provide the same functions and signatures for the usage within the openPOWERLINK stack but allow the connection to an external simulation environment.


Each platform dependent module was implemented for the simulation target.
This specific implementations calls the according functions of the simulation stub.


\subsection{sim folder}
\label{sec:porting_simstub_sim}

\section{Simulated stack}
\label{sec:porting_stack}

\subsection{Interface Implementation}
\label{sec:porting_stack_interface}

\subsection{Multiple instances}
\label{sec:porting_stack_multiinstance}

\section{Simulated nodes}
\label{sec:porting_nodes}

\subsection{Generic node}
\label{sec:porting_nodes_generic}

\subsection{MN}
\label{sec:porting_nodes_mn}

\subsection{CN}
\label{sec:porting_nodes_cn}

\section{Demo application}
\label{sec:porting_demo}

%As described in section \ref{sec:oplk_platform} multiple modules contain platform specific functionalities.
%The openPOWERLINK stack can be built with various configurations, depending on the settings and the compositions configured individually more or less platform specific modules are used.
%These configurations are defined within the different projects in the \emph{proj} folder, as explained in section \ref{sec:oplk_structure_proj}.
%
%\section{Analyze of existing projects}
%\label{sec:porting_projects}
%
%\subsection{liboplkmn}
%\label{sec:porting_projects_liboplkmn}
%
%The \emph{liboplkmn} represents a \emph{MN} library consisting of a single library including the user and kernel space.
%Analyzing the \emph{liboplkmn} for linux results in following modules which must be ported:
%
%\begin{itemize}
%    \item Target
%    \item Timer
%    \item Edrv
%    \item \emph{SDO} via \emph{UDP}
%    \item Trace
%\end{itemize}
%
%The first step for the implementation of the \emph{MN} library for OMNeT++ is the integration in the configuration and build process within the openPOWERLINK stack.
%Therefore an according options file and toolchain file for integration in the \emph{CMAKE} process are created.

