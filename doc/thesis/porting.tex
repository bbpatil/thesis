\chapter{Simulation of openPOWERLINK}
\label{cha:porting}

As described in section \ref{sec:oplk_platform} multiple modules contain platform specific functionalities.
The openPOWERLINK stack can be built with various configurations, depending on the settings and the compositions configured individually more or less platform specific modules are used.
These configurations are defined within the different projects in the \emph{proj} folder, as explained in section \ref{sec:oplk_structure_proj}.

\section{Analyze of existing projects}
\label{cha:porting_projects}

\subsection{liboplkmn}
\label{cha:porting_projects_liboplkmn}

The \emph{liboplkmn} represents a \emph{MN} library consisting of a single library including the user and kernel space.
Analyzing the \emph{liboplkmn} for linux results in following modules which must be ported:

\begin{itemize}
    \item Target
    \item \emph{SDO} via \emph{UDP}
    \item Trace
    \item Timer
    \item CircularBuffer
\end{itemize}

The first step for the implementation of the \emph{MN} library for OMNeT++ is the integration in the configuration and build process within the openPOWERLINK stack.
Therefore an according options file and toolchain file for integration in the \emph{CMAKE} process are created.