\chapter{openPOWERLINK}
\label{cha:oplk}
%TODO general description

\section{Functionalities}
\label{sec:oplk_functionalities}
%TODO introduction in functionality

\subsection{Definitions}
\label{sec:oplk_functionalities_defs}
%TODO definitions like MN, CN, OD, ...

\subsection{Architecture}
\label{sec:oplk_functionalities_arch}
%TODO explain architecture including user, kernel, CAL, AMI, HAL, ...

\section{Structure}
\label{sec:oplk_structure}
%TODO introduction in structure

\subsection{Configuration and build}
\label{sec:oplk_structure_build}

The build tool \emph{CMAKE} is used for dynamic creation of according makefiles using definitions and special sources for platform dependency.

Within the openPOWERLINK folder structure the main \emph{CMAKE} file \emph{CMakeLists.txt} checks the current \emph{CMAKE\_SYSTEM\_NAME} variable and creates the according makefile.
The subfolder cmake the general and platform specific cmake-files are located.
The general files \emph{directories.cmake} and \emph{stackfiles.cmake} are always included in the generation process.

Within \emph{directories.cmake} the location of the source, include, user-, kernel-space, architecture files are defined.
The \emph{stackfiles.cmake} file contains the definitions for specific source files grouped in categories like \emph{event}, \emph{cal} and many more.
The platform specific source files are also defined within this file with the according names.

The global \emph{CMAKE} file loads according to the \emph{CMAKE\_SYSTEM\_NAME} variable the correct system specific options file.
This option file includes options for enabling and disabling part of the openPOWERLINK stack.
For example the generation of the \emph{CN} or \emph{MN} library can be en- or disabled.
Further options are platform specific and provide configuration possibilities for specific usages.

Checking the defined options the according projects are included in the generation and later on the build process.
These projects are located in the \emph{proj} folder.


\subsection{\emph{Proj} folder}
\label{sec:oplk_structure_proj}
The \emph{proj} folder contains different directories for three target systems:

\begin{description}
    \item[generic] contains projects for embedded targets without underlying operating systems.
    Providing implementations of \emph{MN} and \emph{CN} for different targets.
    \item[linux] contains projects for linux operating systems using different technologies (Kernel Module, dirver, ...).
    \item[windows] contains projects for windows operating systems using different technologies (Kernel Module, dirver, ...).
\end{description}

Within a specific target folder the different projects are located.
For example the projects \emph{liboplkmn} and \emph{liboplkcn} exist for windows and linux systems.
The different projects represent either different functionalities or different targets and included technologies.

Specific files for each project are located in each folder, including a \emph{CMakeListst.txt} file.
This is included by the according option files.
Within these \emph{CMakeListst.txt} file the specific source files are gathered and set within variables like \emph{LIB\_SOURCES}.
These variables are used for defining the build targets.
The type, location and additional libraries are also either defined within the project files or the option files.

The build targets mostly are libraries which can be linked to the resulting application.

\section{Platform dependency}
\label{sec:oplk_platform}

As described in section \ref{sec:oplk_structure} the configuration and build process defines the according platform specific implementations and libraries.

For platform specific functionalities the implementation of the openPOWERLINK stack uses common header files with specific implementations.

For defining which implementations are platform specific the \emph{CMAKE} file \emph{stackfile.cmake} is analyzed.

The following modules are implemented platform dependent:

\begin{description}
    \item[oplkinc.h] provides macros for target specific functions like memset, memcopy and so on.
    This header file is located in each project folder.
    \item[Trace] module providing a function for tracing the execution.
    \item[\emph{OBD} configuration] containing specific for functions for storing and restoring of the \emph{OD}.
    The implementation is split in functions regarding file operations and a generic cyclic redundancy check (\emph{CRC}) calculation.
    \item[\emph{SDO} via \emph{UDP}] this implementation contains the transmission of \emph{SDO}s via user datagram protocol (\emph{UDP}).
    The specific implementations provide usage of operating system calls of linux and windows or the usage of a generic header \emph{socketwrapper.h} which allows the implementation for other targets.
    \item[\emph{CAL}] includes different interfaces for the user and kernel space for each containing module (control, \emph{DLL}, \emph{PDO}, error handler, event).
    \item[Timer] providing functions for setting timer.
    \item[Circular buffer] providing the system specific functionalities used within the \emph{circular\_buffer} as creation, allocation, deallocation, locking, connecting and disconnecting.
    \item[Memmap] providing an interface for the memory mapping library used for memory mapped communication in between kernel and user space.
    \item[Target] providing an interface for target specific functions as sleep, interrupt control, tick and implementations of locks and mutexes.
    \item[\emph{AMI}] providing an interface for architecture specific memory functions.
\end{description}

The porting of the openPOWERLINK stack requires the handling and according implementation of the above mentioned modules and functions.
For simulating the openPOWERLINK stack within OMNeT++ the stack must be ported on an OMNeT++ target.
This porting is described in the following chapter \ref{cha:porting}.