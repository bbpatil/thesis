\chapter{OMNeT++}

\label{sec:OMNeT}
OMNeT++ represents a simulation framework written in C++ and is a open source project.
The commercial supported version is OMNEST and provides licensing models, whereas OMNeT++ is only available for academic or non-profit use.
The intention of OMNeT++ is providing infrastructure for writing simulations for various fields.
Especially in the field of network simulations OMNeT++ is widely used due the big number of available libraries.

The architecture and topology of simulations is built with the different OMNeT++ components.

\section{Components}
Within an OMNeT++ simulation different components are used to represent the simulated system.
Each component is described with a \emph{network description} (NED) file and can be enhanced with C++ code.

The outermost component is a network, which consists of other components like modules and channels.
The simulated topology and the connections between modules are defined within the network.

A simple module is the smallest part within a simulated OMNeT++ hierarchy and represent a functional unit.
For this functional unit the behavior for handling messages, the possible connections and additional parameters can be defined.

The possible connections of modules are represented by gates, which can be connected to a channel or directly to other gates.
Multiple simple modules can be connected via channels and condensed to a compound module.
Such compound modules can be used in the same way as simple modules, but represent bigger functional groups.

The connections in between modules can be realized in different ways.
An direct connection of two gates transports the transmitted messages immediately.
For applying transportation parameters (e.g. delay, latency, jitter) the connection can be established with a channel.
Such a channel can implement a simple delayed transportation or complex customized functionality.


An example network including simple modules connected to a compound module is shown in figure \ref{fig:OMNeTComponents}.
Each module shown in figure \ref{fig:OMNeTComponents} defines two gates which can ether be an input, output or bidirectional gate.

\begin{figure}
    \centering
    \includegraphics[width=0.9\columnwidth]{OMNeTComponents.eps}
    \caption{OMNeT++ components in an example network}
    \label{fig:OMNeTComponents}
\end{figure}

Each custom module and channel is defined in the \emph{NED} language.
Specific functionality for custom components is implemented in the according C++ code.
This assignment is usually done with identical names of the \emph{NED} file and the C++ code.
Examples and the combination of \emph{NED} files and C++ code is shown in \cite[chapter 3, chapter 4]{omnet_manual}.
The components can embed any functionality implemented in C/C++.
The usage of external libraries or language features is not limited, but must be used with care due to the effect on simulation performance.


\section{Messages}
Transmitted data are encapsulated in another component called messages.
Messages are a fundamental component of a OMNeT++ simulation as they does not only transport data they can also represent functional messages like jobs, events or tasks.
The meaning behind a message is depending on the written simulation and the simulated system.

These messages can also be customized for holding a specific set of data like a protocol header, checksum, etc. or other specific data.
The existing message class \emph{cMessage} and its derived specialization \emph{cPacket} provide different members and functions which can be used for simulations.
These include control information, type information, time stamp, etc. and are included for making developing a simulation easier.
Adding a few simple datafields to a message can be ether done by subclassing \emph{cMessage} or \emph{cPacket} or using the \emph{NED} syntax.
By defining a custom message using \emph{NED} a customized subclass will be generated by the simulation and can be normally used as Message.

Any module can send a message via a connected channel for this sent message a time is defined.
This time describes the moment when the message should be sent to the channel.
The mechanism of messages is also used for implementing timeouts, timers, etc. by sending a specific message to the current module, this is a so called \emph{self-message}.
These \emph{self-message} is handled by the same function as any other message coming from other modules.
For the identification of \emph{self-messages} there is a built in function available.
More information about messages and the possibilities of customized messages are given in \cite[chapter 5]{omnet_manual}.

Each message sent ether by another module or by the current module itself represents an event for the simulation with an according time, at which this event should happen, or the message should be delivered/received.
The execution and the handling of such created events is done by the simulation core and defines the execution order and the performance of the simulation.
Handling these events can be done in various ways and define the type of implemented simulation.
These simulations based on events are event based discrete simulations, the definitions of this type and the explanation of different simulation types are shown in the next section.

\section{Results}
\emph{OMNeT++} provides multiple functionalities for recording and saving different results of a simulation.


\subsection{Simulation library}
The traditional way to record simulation results is using the simulation library functions to record vectors or scalars of different objects.

Since OMNeT++ 4.1 the newer strategies using \emph{signals} and \emph{statistics} are available and provide alternative methods for results recording.

\subsection{Signals}
Signals provide a functionality for a communication between modules which are not directly connected via their gates.
The usage of signals follows the \emph{publisher/subscriber} principle, i.e. modules can register callback objects to a specific signal.
If a new value for the signal is emitted all registered callback objects are notified.

Signals are defined in the \emph{NED} file of the according module or channel and can be accessed via the library call \emph{signal} and the defined name of the signal. %TODO check


\subsection{Statistics}



\section{Running an OMNeT++ simulation}
A OMNeT++ simulation is defined by a configuration file (\emph{.ini}-file).

\subsection{Configuration}
This configuration file can be given by a command line parameter, by default the \emph{omnetpp.ini} file is used.
The configuration files includes every information and parameter regarding the simulation.

The simulated network must be defined, all other parameters are optional.



A simulation application developed with OMNeT++ can be run in different ways using different simulation environments.

\subsection{Tkenv}
OMNeT++ provides the graphical environment \emph{Tkenv} for developing simulations.
Various possibilities for graphical representation are useful for demographic purposes.
This environment is based on the opensource framework \emph{Eclipse} and provides the integrated development environment (\emph{IDE}) for OMNeT++ simulations.

\subsection{Cmdenv}
The other environment for running \emph{OMNeT++} simulations is \emph{Cmdenv}, which represents a command line interface.
Using this environment no graphical user interface is shown, or will be updated.
This simulation method is recommended for batch simulations or running simulations with an increased runtime.