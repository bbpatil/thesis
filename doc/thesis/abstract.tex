\chapter{Abstract}

The fields of simulation and emulation are steadily gaining importance for the development of embedded systems.
This is caused by the raising requirement of comprehensive testing.
Various fields of application as the automotive industry demand a high level of testing.

This leads to the urge of implementing a simulation or emulation for given embedded systems or new systems under development.
The field of simulation shows a big number of different systems and frameworks.
Each of them provides various configuration possibilities.
This big number of different possibilities and strategies for developing a system can be keeping developer from making the decision to develop a customized simulation.
OMNeT++ embodies a highly flexible simulation framework for various applications.

This thesis analyzes the properties of OMNeT++ and shows the possibilities for configuration and modification for achieving specific goals.
Different performance measurements are made for analyzing a test simulation in terms of runtime, processed events and real-time capabilities.
An optimized design for achieving good performances is found.

The functionalities and possibilities for real-time simulation and emulation portrays an interesting field for the developing and testing of embedded systems.

The Open Source implementation openPOWERLINK provides the possibility to analyze and modify an existing real-time communication system.
openPOWERLINK implements the POWERLINK protocol which is widely used in the field of automation and real-time communication.
Developing a simulation embedding a given system like the openPOWERLINK stack brings up interesting questions regarding multiple instances.
A solution is found and a POWERLINK network consisting of openPOWERLINK nodes can be simulated.


%TODO: (re)write abstract
%% imported from paper
%OMNeT++ embodies a framework for simulations. This includes different functionalities for communication in between modules and components written in C++.
%
%
%
%OMNeT++ provides the possibility for running a simulation in real time i.e. every simulated second is processed within a real second.
%Such a real time simulation attempts to process the simulation with real world timings and delays.
%The possibility to run simulated software with real timings allows the field of emulation and the connection of simulated components with real hardware or software.
%Such a real time simulation and its limits regarding possible timings depends on the used host system.
%Implementing the simulation for a given software system using OMNeT++ can be achieved with different designs and various numbers of modules and transmitted messages.
%These factors will impact the efficiency of the real time simulation and achieved timings.
%
%This paper investigates the functionality of OMNeT++ regarding simulation and real time simulation.
%TODO: write german abstract