\chapter{Simulation}
\label{cha:simulation}

A Simulation attempts to replicate and forecast the behavior of real world systems.
The importance of simulation is increasing steadily, due to increasing complexity and more requirements for systems.
Especially the development and testing of real-time communication systems can be improved essentially by using simulation and emulation techniques.

Different types of simulation are applicable for different types of simulated systems and aimed results.
Differences are shown in the processing of the simulated systems and in the handling of simulation time.
\cite[section 1.2]{mchaney2009understanding}

\section{Continuous simulation}
\label{sec:simulation_cont}
Continuous simulations handles uninterrupted values over a simulated time range.
This behavior may be determined by equations describing the system.
For correct simulation of a continuous system the model must be executed for the whole simulated time range. \cite[section 1.2.1]{mchaney2009understanding}

Simulations in the field of real-time communication is based on discrete events at specific points in time, e.g. receiving of data.
The continuous processing of occasions in between events is often not necessary.
Therefore an discrete event simulation (\emph{DES}) is more applicable.

\section{Discrete event simulation}
\label{sec:simulation_event}
This type of simulation is based on processing discrete events.
During the processing of events the simulation time does not advance and the required time is not considered in simulation time.
The simulation time is advancing with multiple processed events and their defined point in simulation time for execution.
In between two consecutive events no processing or changes of the system state is happening. \cite[chapter 1]{matloff_introduction_2008}

The assumption is that no, for simulation relevant, events and changes are happening in between two consecutive events.
This exclusion is done by the implementation of the simulated model.
Simulating a real world system therefore requires a filtering of interesting occasions. \cite[section 4.1.1]{omnet_manual}

The implementation of a \emph{DES} can be done in various ways using different strategies or paradigms. \cite[chapter 2]{matloff_introduction_2008}
Using OMNeT++ the \emph{Event-Oriented Paradigm} and the \emph{Process Oriented Paradigm} are applicable and can be achieved as described in section \ref{sec:omnet_components_modules}.

%TODO: enhance and rework existing texts

%By default simulations written with OMNeT++ are discrete event based simulations.
%More information about the simulation types available in OMNeT++ are given in \cite[section 4.1]{omnet_manual}.

%TODO: check if useful
%%%%%%%%%% imported from omnet chapter %%%%%%%%%%%%%%%%%%%%%%%%%
% Handling these events can be done in various ways and define the type of the implemented simulation.
% Simulations based on events are event based discrete simulations.
% The definitions of this type and the explan
% Handling these events can be done in various ways and define the type of implemented simulation.
% These simulations based on events are event based discrete simulations, the definitions of this type and the explanation of different simulation types are shown in the next section.



\section{Real time simulation} %TODO: rework and enhance references
\label{sec:simulation_real_time}
Real time simulations change the meaning of simulation time.
The simulated events should be executed at the correct time to match the real-time.
In this context the real-time means the real world time, cpu time, or wall time, i.e. the time which passes for the real world during the execution of the simulation.
This type of simulation is not possible for every simulated system as the limits are defined by the execution speed of the host system.
Achieving this match of simulation time and real time is strongly depending on the targeted time resolutions and time spans between events.
The needed time for calculating new events and reacting to incoming events is defined by the executed functions and is therefore defined by the simulated system.



\section{Simulation with OMNeT++}
\label{sec:simulation_omnet}
Within OMNeT++ the calculated point in time is called \emph{arrival time}.
Events are created by modules and then inserted in the so called future event structure (\emph{FES}).
The simulation core executes all events within the \emph{FES} at the according simulation time.

The main part of the simulation core of OMNeT++ which controls the event handling is the scheduler.
This scheduler accesses the \emph{FES} and chooses the next event to be handled by the simulation.
The class \emph{cScheduler} represents the interface which is required for an event scheduler usable in OMNeT++.
By default the derived class \emph{cSequentialScheduler} is used.
This scheduler implements the default discrete event based simulation and handles the events according to their execution time, scheduling priority and scheduled time.
The scheduling priority provides a mechanism for controlling the execution order of multiple events at the same time.
The functionality and the ordering of the executed events by the \emph{cSequentialScheduler} are explained in \cite[section 4.1]{omnet_manual}.

Approaching the fields of emulation and \emph{HiL} the discrete event based simulation is not applicable.
For such applications the type of real time simulation is required.
\\



% OMNeT
Lightweight simple modules with plain functions can be simulated with faster event frequencies than compound modules consisting of multiple modules with complex behavior.

Approaches to executing the type of real-time simulations are implemented in the \emph{cRealTimeScheduler} within OMNeT++.
This scheduler executes the events according to their planned arrival time.
The arrival time of the next event is compared with the current real time.
When the simulation is ahead of the real time, the simulation is paused for the remaining time.
The \emph{cRealTimeScheduler} waits in hard-coded 100 ms chunks for achieving a responsive simulation.
For emulations and \emph{HiL} this concept is not applicable, because the communication with real components does not allow a sleep time.
The OMNeT++ sample \emph{sockets} demonstrates this problem and a possible solution with a custom scheduler implementing the \emph{cScheduler} class.
This custom scheduler named \emph{SocketRTScheduler} listens to the network interface during the times when the simulation has to wait until the next event should be executed.
This allows the receiving of packets from real clients and the connection of the simulated components with real ones.
The implementation of \emph{SocketRTScheduler} is not fully optimized and is intended to show the possibility of emulation and \emph{HiL} using OMNeT++.

Handling a simulation which is faster than the real world system can be done in various ways as demonstrated by \emph{cRealTimeScheduler} and \emph{SocketRTScheduler}.
The implementation of the scheduler and a description of their functionality can be found in the OMNeT++ sources and samples or in the API reference \cite{omnet_api}.

If the simulation is lagging behind the real-time, the scheduler must try to speed up the simulation and catch up to the real time.
The \emph{cRealTimeScheduler} executes the next events immediately and therefore skipping sleep times.
With this behavior the task of catching up to the real-time becomes very difficult for complex simulations with tight timings.
If the simulation lags constantly behind the real world using the \emph{cRealTimeScheduler} it becomes a discrete event based simulation.

The achieved simulation time can be defined by the performance ratio which can be displayed during simulation.
This ratio represents the simulated seconds per real time seconds.
A lagging simulation is defined by a performance ratio of less than one and simulation which simulates faster as the real time shows a ratio greater one.
The goal of a real time simulation is a constant ratio of one.
The process of catching up a lagging simulation to achieve a performance ratio of one can also influence other timings.
Therefore the variation of delays (jitter) increases when the simulation lags temporarily.
For emulations or the fields of \emph{HiL} an increased jitter for a signal can be very critical and must be analyzed carefully.

The quality of the real-time simulation is strongly depending on the simulated system and its composition.
Therefore the ideal results can be achieved by writing a custom real time scheduler which is optimized for the specific simulations.
The sample scheduler provided by \emph{cRealTimeScheduler} and \emph{SocketRTScheduler} can lead to the correct strategy of implementing an optimized scheduler.

The host machine for the simulation and its components affect the achieved simulation times.
The dependencies of the host system and the results of existing researches is shown in section \ref{sec:parallel_sim_requriements}.

Developing the simulation of a given system results in the situation of existing code.
This code must be encapsulated in modules be executed depending on incoming messages and therefore creating new message for sending.
Given systems can be designed in various hierarchies in sight of number of modules and complexity of simple modules.
The different designs and their effect on real time simulation is shown in the chapter \ref{cha:design}.

