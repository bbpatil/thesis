\chapter{Simulation}
\label{cha:simulation}

%%%%%%%%%% imported from omnet chapter %%%%%%%%%%%%%%%%%%%%%%%%%
% Handling these events can be done in various ways and define the type of implemented simulation.
% These simulations based on events are event based discrete simulations, the definitions of this type and the explan
% Handling these events can be done in various ways and define the type of implemented simulation.
% These simulations based on events are event based discrete simulations, the definitions of this type and the explanation of different simulation types are shown in the next section.



Simulation can be done in different ways. The general categories are \emph{discrete} and \emph{continuous} simulations.
The difference is shown in the processing of the simulated systems and the handling of simulation time.

For continuous simulations the state of the simulated system can change at any time, therefore the complete simulated time range has to be executed.
This executions requires the definitions of a resolution and the calculation of each moment within the time range.

This requirement is not necessary for discrete simulations, because these simulations are event based.
For each event a time is specified at which the event has to be executed or handled.
For discrete simulations the system state is constant between two consecutive events and therefore this time can be skipped in simulation time.
The calculated moments within the simulation time are defined by the schedules events and their execution time.

Simulations written with OMNeT++ are by default discrete event based simulations.
More information about the simulation types available in OMNeT++ are given in \cite[section 4.1]{omnet_manual}.

\section{Event simulation}
\label{sec:simulation_event}
Such simulations handle given events at the defined point in the simulation time.
The execution of the event takes no simulation time.
Each event processed in the simulation is handled at the exact simulation time which is defined individually.
This type of simulation enables simulating various events even within a very short simulated time range.

Within OMNeT++ this time is called \emph{arrival time}.
Each message contains the \emph{arrival time} and the \emph{sending time}, both are time points in the simulation time.
The \emph{sending time} is used for calculation of transmissions.
For the execution and handling of an event the \emph{arrival time} is more important.
Events are created by modules and then inserted in the so called future event structure (\emph{FES}).
The simulation core executes all events within the \emph{FES} at the according simulation time.

The main part of the simulation core of OMNeT++ which controls the event handling is the scheduler.
This scheduler accesses the \emph{FES} and chooses the next event to be handled by the simulation.
The class \emph{cScheduler} represents the interface which is required for a event scheduler usable in OMNeT++.
By default the derived class \emph{cSequentialScheduler} is used.
This scheduler implements the default discrete event based simulation and handles the events according to their execution time, scheduling priority and scheduled time.
The scheduling priority provides a mechanism for controlling the execution order of multiple events at the same time.
The functionality and the ordering of executed event by the \emph{cSequentialScheduler} are explained in \cite[section 4.1]{omnet_manual}.

Approaching the field of emulation and \emph{HiL} the discrete event based simulation is unusable.
For such applications the type of real time simulation is required.

\section{Real time simulation}
\label{sec:simulation_real_time}
Real time simulations change the meaning of simulation time.
The simulated events should be executed at the correct time to match the real time.
In this context \emph{the real time} means the real world time, cpu time, or wall time, i.e. the time which passes for the real world during the execution of the simulation.
This type of simulation is not possible for every simulated systems as the limits are defined by the execution speed of the host system.
Achieving this match of simulation time and real time is strongly depending on the targeted time resolutions and time spans between events.
The needed time for calculating new events and reacting to incoming events is defined by the executed functions and is therefore defined by the simulated system.
Lightweight simple modules with plain functions can be simulated with faster event frequencies than compound modules consisting of multiple modules with complex behavior.

Approaches for achieving this type of simulations are implemented in the \emph{cRealTimeScheduler} within OMNeT++.
This Scheduler executes the events according to their planned arrival time.
The arrival time of the next event is compared with the current real time.
When the simulation is ahead of the real time, the simulation is paused for the remaining time.
The \emph{cRealTimeScheduler} waits in hard-coded 100 ms chunks for achieving a responsive simulation.
For emulations and \emph{HiL} this concept is not applicable, because the communication with real components does not allow a sleep time.
The OMNeT++ sample \emph{sockets} demonstrate this problem and a possible solution with a custom scheduler implementing the \emph{cScheduler} class.
This custom scheduler named \emph{SocketRTScheduler} listens to the network interface during the times when the simulation has to wait until the next event should be executed.
This allows the receiving of packets from real clients and the connection of the simulated components with real ones.
The implementation of \emph{SocketRTScheduler} is not fully optimized and is intended to show the possibility of emulation and \emph{HiL} using OMNeT++.

Handling the difference for a simulation which is faster than the real world system can be done in various ways as demonstrated by \emph{cRealTimeScheduler} and \emph{SocketRTScheduler}.
The implementation of the scheduler and a description of their functionality can be found in the OMNeT++ sources and samples or in the API reference \cite{omnet_api}.

Is the simulation lagging behind the real time, the scheduler must try to speed up the simulation and catch up to the real time.
The \emph{cRealTimeScheduler} executes the next events immediately and therefore skipping sleep times.
With this behavior the task of catching up to the real time becomes very difficult for complex simulations with tight timings.
Lags the simulation constantly behind the real world using the \emph{cRealTimeScheduler} it becomes discrete event based.

The achieved simulation time can be defined by the performance ratio which can be shown during simulation.
This ratio represents the simulated seconds per real time seconds.
A lagging simulation is defined by a performance ration of less than 1 and simulation which simulates faster as the real time shows a ratio of greater than one.
The goal of a real time simulation is a constant ratio of 1.
The process of catching up by a lagging simulation and achieving a performance ratio of 1 can also influence other timings.
Therefore the variation of delays (jitter) increases when the simulation lags temporarily.
For emulations and the field of \emph{HiL} a increased jitter for a signal can be very critical and must be analyzed carefully.

The quality of the real time simulation is strongly depending on the simulated system and its composition.
Therefore the ideal results can be achieved by writing a custom real time scheduler which is optimized for the specific simulations.
The sample scheduler provided by \emph{cRealTimeScheduler} and \emph{SocketRTScheduler} can lead to the correct strategy of implementing an optimized scheduler.

The host machine for the simulation and its components affect the achieved simulation times.
The dependencies of the host system and the result of existing researches is shown in section \ref{sec:SystemDependencies}. %TODO update ref

Developing the simulation of a given system results in the situation of many provided code which should be executed depending on incoming messages and therefore creating new message for sending.
Given systems can be designed in various hierarchies in sight of number of modules and complexity of simple modules.
The different designs and their effect on real time simulation is shown in the next section.

