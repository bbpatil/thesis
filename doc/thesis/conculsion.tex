\chapter{Conclusion}
\label{cha:conclusion}

\section{Future simulations}
\label{sec:conclusion_futuresim}
The developed OMNeT++ modules allow the simulation of a simple POWERLINK network consisting of a openPOWERLINK \emph{MN} and multiple \emph{CNs}.
This simulation was developed for optimal performance using a monolithic design.
Alternative simulations can be implemented using the basic strategies and reusable functionalities represented by base classes, helper classes and interfaces.
For more insight a more modular design could be implemented by connecting the \emph{CAL} modules with the simulation by implementing according target specific implementations redirecting the communicated calls and data into the simulation environment.

The simulated network contains a direct connection of the different nodes.
This represents an optimal behavior and results in an ideal performance.
For simulating the behavior of openPOWERLINK nodes in a network including real devices this connection would have to be changed.
Existing functionality from the INET library regarding Ethernet and UDP can be used.

\section{Emulation}
\label{sec:conclusion_emulation}



\section{Outlook}
\label{sec:conclusion_outlook}

%TODO: rewrite conclusion

% imported from paper
OMNeT++ provides many different features for developing various simulations.
The built in mechanisms for real time simulations represent a possibility for emulations and the field of \emph{HiL}.
Due to the simulated system's effect on the achievable capabilities, each system must be analyzed individually.
Such an analyze can also be used for the implementation of an optimized scheduler which can lead to improved performances for specific applications.


The general increased overhead of a modular design can limit the achievable timings of a real time simulation, but executed with parallelization this can lead to a improved performance.
Is the efficient parallel execution of a simulated system not possible due to a too high level of dependencies between modules a monolithic design should be considered for decreasing simulation overhead


The open source project OMNeT++ provides the possibility of analyzing the simulation and developing optimized and customized solutions.
These solutions provide remarkable opportunities for emulation and the field of \emph{HiL} and are therefore useful for different applications.