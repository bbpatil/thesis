% !TeX spellcheck = de_AT_frami

\chapter{Kurzfassung}
\begin{german}
Die Themengebiete der Simulation und Emulation gewinnen stetig an Bedeutung für die Entwicklung von eingebetteten Systemen.
Dies zeigt sich in der steigenden Nachfrage von umfangreichen Tests.
Eine Vielzahl von Anwendungsgebieten erfordern einen hohen Grad an Tests.

Dieser Trend führt zu einem steigendem Interesse an der Entwicklung einer Simulation oder Emulation für bestehende oder neu entwickelte Systeme.
Es existiert eine große Anzahl von verschiedenen Systemen zur Entwicklung von Simulationen die oftmals vielseitige Konfigurationsmöglichkeiten bieten.
Dieser Überfluss an verschiedenen Möglichkeiten eine Simulation zu entwickeln, kann Entwickler von der Entscheidung abhalten, eine Simulation für ihren Nutzen zu entwickeln.
OMNeT++ stellt ein sehr flexibles System zur Entwicklung von Simulationen für verschiedenste Anwendungen dar.

\begin{sloppypar}
Diese Arbeit analysiert den Aufbau und die Eigenschaften von OMNeT++ und zeigt die Möglichkeiten zur Konfiguration und Anpassung.
Es werden unterschiedliche Leistungsmessungen durchgeführt, die eine Testsimulation im Bezug auf Laufzeit, verarbeiteten Ereignissen und Echtzeiteigenschaften überprüft.
Es wurde ein optimierter Aufbau der Simulation gefunden, welcher zu einem verbesserten Ergebnis der Leistung führt.
Die vorhandenen Funktionen und die Möglichkeiten zur Entwicklung einer Echtzeitsimulation und Emulation stellt eine interessante Methode zum Testen und Entwickeln von eingebetteten Systemen dar.
Die Open Source Implementierung openPOWERLINK bietet die Möglichkeit, ein bestehendes Echtzeitkommunikationssystem zu analysieren und für eine Simulation anzupassen.
openPOWERLINK implementiert das weit verbreitete POWERLINK Protokoll, welches häufig im Bereich der Industrieautomation und Echtzeitkommunikation eingesetzt wird.
Bei der Entwicklung einer Simulation, die ein gegebenes System, wie die openPOWERLINK Implementierung, einbindet, werden Fragen bezüglich Mehrfachinstanzen aufgeworfen.
Es wurde eine Lösung für diese Problematik gefunden und anschließend eine Simulation entwickelt, die ein aus openPOWERLINK Knoten bestehendes POWERLINK Netzwerk nachstellt.
\end{sloppypar}
    
\end{german}
