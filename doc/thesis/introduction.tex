\chapter{Introduction}
\label{cha:introduction}
This thesis is intended to analyze the simulation framework \emph{OMNeT++} \cite{omnet_manual} and the possibilities for real-time simulation and emulation of a real time communication network.
Furthermore a simulation of the real-time communication protocol POWERLINK and its open source implementation \mbox{openPOWERLINK} \cite{openpowerlink} is implemented. 
This simulation is also analyzed regarding real-time simulation and emulation capabilities.

\section{Motivation}
% needs and requirements for embedded systems
The fields of embedded systems and real-time communication entail critical requirements for timings and deterministic behavior.
The development of systems meeting those requirements can be very difficult due to the lack of possibilities for sufficient testing.
A variable testing environment is easing the development and verification of such systems.
The correct replication of complex scenarios and special operating conditions is not always possible.
Therefore the need for simulation is gaining importance with increasing complexity of embedded systems.

% real world simulation
Extended functionalities of simulation frameworks allows the usage as real-time simulation.
By simulating a system in real time, i.e. the simulated time passes according to the real (world) time, these simulations can be used in the fields of emulation and hardware in the loop (\emph{HiL}).
These fields allow testing scenarios built with a variable simulated environment
These fields are necessary for testing embedded systems, due to the increased possibilities of testing scenarios and setups.

% benefits for development
The development benefits from the enhanced possibilities of testing the system under development.
Improved testing increases the chance of detecting a mistake before introducing it in the system.

%TODO extend


\section{Content}
%TODO

% first chapters fundamentals and basics of OMNeT++, simulation, emulation, simulated designs, parallel simulation

% development of a example network for design analyze and discussion of results

% fundamentals of openPOWERLINK with focus on platform dependency and porting possibilities

% approach to simulate openPOWERLINK in OMNeT++

% resulting simulation and maybe performance

% conclusion

% appendix

\section{Purpose}
%TODO