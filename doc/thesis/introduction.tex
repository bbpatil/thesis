\chapter{Introduction}
\label{cha:introduction}
This thesis is intended to analyze the simulation framework \emph{OMNeT++} and the possibilities for real-time simulation and emulation of a real time communication network.
Furthermore the simulation, real-time simulation and emulation of the real-time communication network POWERLINK with its implementation openPOWERLINK is implemented and analyzed.

\section{Motivation}
In the field of embedded systems the testing of developed applications and systems may bring up different problems due to limited access and resources.
Due to the increased difficulties for the testing of running systems the need for simulations is getting stronger.

During the development of various systems a existing simulation is useful for testing newly developed features.

Extended functionalities of simulation frameworks allows the usage as real time simulation.
By simulating a system in real time, i.e. the simulated time passes according to the real (world) time, the fields of emulation and hardware in the loop (\emph{HiL}).
These fields are necessary for testing embedded systems, due to the increased possibilities of testing scenarios.

% benefits for development
The development benefits from the enhanced possibilities of testing the system under development.
Improved testing increases the chance of detecting a mistake before introducing it in the system.

%TODO extend


\section{Content}
%TODO

\section{Purpose}
%TODO

